\documentclass{scrartcl}
\usepackage{leopackages}
\usepackage{leoshortcuts}

\newcommand*{\sectionpostamble}{}
\newcommand*{\teacher}[1]{%
      \def\sectionpostamble{#1}%
}

\usepackage{titlesec}
\titleformat{\section}
  {\normalfont\Large\bfseries}{\thesection}{1em}{}
  [\normalfont\small\itshape\raggedleft\sectionpostamble
  \global\let\sectionpostamble\relax]

\usepackage{blindtext}


\title{Project 1}
\subtitle{'Bounce' N-body solver}
\author{Leopold Talirz}

\begin{document}

\maketitle
\tableofcontents

\chapter{Class outline}
The two main tasks of an N-body solver are
\begin{enumerate}
    \item to calculate the forces for a given point in phase space and 
    \item to update the phase space coordinates using the forces as well as
        the kinetic energy.
\end{enumerate}

Since our solver should be able to treat different kinds of interactions,
the forces will be modularized in two levels of abstraction: 
First the kind of potential (two, three, \ldots N-body) and then the
actual form of the potential.

Furthermore, we will need to compile the code on CPU and GPU architectures. 
We will use preprocessor instructions to swtich between the different
implementations for the different architectures (?).




\end{document}
